\section{Risultati degli algoritmi}
Questa sezione risponderà alla Domanda 1: verranno riportati i grafici del tempo impiegato in funzione della dimensione del grafo, le performance ed il costo dei MST calcolati dagli algoritmi \texttt{Prim}, \texttt{Kruskal} e \texttt{NaiveKruskal}.\vspace{-10pt}

\subsection{Specifiche Hardware dei calcolatori utilizzati}
Dato il considerevole divario di prestazioni ottenute dalle due diverse macchine, abbiamo ritenuto opportuno riportare le differenti componenti hardware su cui abbiamo lavorato.
\begin{center}
	\begin{longtable}{ r | c | c } %\hline
	\multicolumn{1}{c|}{\textbf{Caratteristica}} &\textbf{PC di Nicola}&\textbf{PC di Federico}\\ \hline 
	\endfirsthead
	\rowcolor{white}
	\multicolumn{3}{|r|}{\textit{-- continuazione da pagina precedente}} \\ \hline 
	\endhead
	\hline
	\rowcolor{white} 
	\multicolumn{3}{|r|}{{\textit{-- continua a pagina successiva}}} \\
	\endfoot
	\endlastfoot
	%table input should begin here
	Nome processore & Intel i5-7300HQ & Intel i7-8750H \\
	Numero core & 4 & 6\\
	Numero thread & 4 & 12 \\
	Range velocità di clock [GHz] & 2.50 - 3.50 & 2.20 - 4.10\\
	Dimensione cache L1 [KiB] & 256 & 384\\
	Dimensione cache L2 [MiB] & 1 & 1.5\\
	Dimensione cache L3 [MiB] & 6 & 9\\	
	Dimensione RAM [GiB] & 8 & 31.2\\  \hline
	\caption{Specifiche dei calcolatori utilizzati.\\ Fonte: \url{http://intel.com}.}
	\end{longtable}\vspace{-30pt}
\end{center} \vspace{-30pt}

\subsection{Prim}
L'algoritmo non presenta variazioni nell'implementazione rispetto all'algoritmo mostrato a lezione, dunque possiede una complessità di \comp{n\log n}.
\image{0.741}{prim_g}{Performance dell'algoritmo Prim sui due sistemi}

L'algoritmo è decisamente performante per grafi fino a 10K nodi, successivamente inizia ad essere relativamente lento per grafi da 80K nodi, impiegando $\sim$1 minuto e mezzo, fino ad arrivare a grafi con 100K nodi impiegando $\sim$4 minuti, nel caso della macchina di Federico Brian.\acapo

Di seguito sono riportati il tempo computazionale ed i pesi dei MST calcolati.

\begin{center}
	\begin{longtable}{|c|c|c|c|c|}
		\hline
		\textbf{N.} & \textbf{Graph Size (nodes)} & \textbf{Time Federico (s)} & \textbf{Time Nicola (s)} & \textbf{MST cost} \\
		\hline
		\endfirsthead
		\multicolumn{5}{|c|}%
		{\tablename\ \thetable\ -- \textit{continuazione della pagina precedente}} \\
		\hline
		\textbf{N.} & \textbf{Graph Size (nodes)} & \textbf{Time Federico (s)} & \textbf{Time Nicola (s)} & \textbf{MST cost} \\
		\hline
		\endhead
		\hline \multicolumn{5}{|r|}{\textit{Continua nella pagina seguente}} \\
		\endfoot
%		\hline
		\endlastfoot
		1 & 10 & 0.001516935 & 0.005691 & 29316\\
2 & 10 & 0.000139756 & 0.0002781 & 2126\\
3 & 10 & 0.000149669 & 0.0002143 & -44765\\
4 & 10 & 0.000115304 & 0.0001572 & 20360\\
\hline
5 & 20 & 0.000201304 & 0.002932 & -32021\\
6 & 20 & 0.000148581 & 0.0002687 & 18596\\
7 & 20 & 0.000130301 & 0.0005665 & -42560\\
8 & 20 & 9.9989e-05 & 0.000397 & -37205\\
\hline
9 & 40 & 0.000255545 & 0.0004462 & -122078\\
10 & 40 & 0.000227131 & 0.001578 & -37021\\
11 & 40 & 0.000223172 & 0.0012624 & -79570\\
12 & 40 & 0.000190513 & 0.0004704 & -79741\\
\hline
13 & 80 & 0.000461337 & 0.0022566 & -139926\\
14 & 80 & 0.000358958 & 0.0004229 & -211345\\
15 & 80 & 0.000312896 & 0.0004965 & -110571\\
16 & 80 & 0.000331155 & 0.0055763 & -233320\\
\hline
17 & 100 & 0.000222887 & 0.0003762 & -141960\\
18 & 100 & 0.000186166 & 0.0003986 & -271743\\
19 & 100 & 0.00020321 & 0.0053806 & -288906\\
20 & 100 & 0.000191939 & 0.0003545 & -232178\\
\hline
21 & 200 & 0.000556103 & 0.0008466 & -510185\\
22 & 200 & 0.000539168 & 0.0012841 & -515136\\
23 & 200 & 0.000626964 & 0.0007046 & -444357\\
24 & 200 & 0.000466611 & 0.0007651 & -393278\\
\hline
25 & 400 & 0.001343472 & 0.0081984 & -1122919\\
26 & 400 & 0.001259603 & 0.0062876 & -788168\\
27 & 400 & 0.001300368 & 0.0024222 & -895704\\
28 & 400 & 0.001263466 & 0.0022616 & -733645\\
\hline
29 & 800 & 0.004210081 & 0.0156534 & -1541291\\
30 & 800 & 0.004169337 & 0.0170512 & -1578294\\
31 & 800 & 0.004372859 & 0.0111666 & -1675534\\
32 & 800 & 0.004226525 & 0.012685 & -1652119\\
\hline
33 & 1k & 0.006369881 & 0.0271609 & -2091110\\
34 & 1k & 0.00644721 & 0.0261404 & -1934208\\
35 & 1k & 0.006475871 & 0.0232982 & -2229428\\
36 & 1k & 0.006468983 & 0.0134004 & -2359192\\
\hline
37 & 2k & 0.025558904 & 0.0526213 & -4811598\\
38 & 2k & 0.02420279 & 0.0500484 & -4739387\\
39 & 2k & 0.024796339 & 0.0477902 & -4717250\\
40 & 2k & 0.025268448 & 0.0457845 & -4537267\\
\hline
41 & 4k & 0.103823489 & 0.1973502 & -8722212\\
42 & 4k & 0.103343875 & 0.2438554 & -9314968\\
43 & 4k & 0.112767785 & 0.2912372 & -9845767\\
44 & 4k & 0.102867612 & 0.2805648 & -8681447\\
\hline
45 & 8k & 0.466010458 & 1.1943197 & -17844628\\
46 & 8k & 0.457934774 & 1.4036635 & -18800966\\
47 & 8k & 0.45301853 & 1.025199 & -18741474\\
48 & 8k & 0.463987517 & 1.5333734 & -18190442\\
\hline
49 & 10k & 0.745876695 & 3.3735423 & -22086729\\
50 & 10k & 0.748228046 & 1.9356491 & -22338561\\
51 & 10k & 0.752410272 & 1.6555519 & -22581384\\
52 & 10k & 0.741559471 & 1.7339627 & -22606313\\
\hline
53 & 20k & 3.366825299 & 7.2241058 & -45978687\\
54 & 20k & 3.317811526 & 9.442905599 & -45195405\\
55 & 20k & 3.31394422 & 9.409197101 & -47854708\\
56 & 20k & 3.35028821 & 10.1661279 & -46420311\\
\hline
57 & 40k & 14.907811327 & 58.2513275 & -92003321\\
58 & 40k & 15.069201141 & 56.572616 & -94397064\\
59 & 40k & 16.985234135 & 63.9437065 & -88783643\\
60 & 40k & 17.136569027 & 52.5889188 & -93017025\\
\hline
61 & 80k & 99.832574806 & 305.1820434 & -186834082\\
62 & 80k & 95.712610549 & 292.0056872 & -185997521\\
63 & 80k & 87.760898325 & 300.2967593 & -182065015\\
64 & 80k & 81.084580784 & 317.9336533 & -180803872\\
\hline
65 & 100k & 170.88136533 & 505.2773482 & -230698391\\
66 & 100k & 180.640790693 & 590.8584786 & -230168572\\
67 & 100k & 190.185899105 & 589.5148767 & -231393935\\
68 & 100k & 232.491690129 & 594.4183702 & -231011693\\ \hline
		\caption{Risultati dell'algoritmo di Prim} \\
	\end{longtable}
\end{center}
\input{./res/sections/Naivekruskal}
\input{./res/sections/Kruskal}
