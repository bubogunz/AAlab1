\section{Conclusioni} \label{conc}
In questa sezione risponderemo alla Domanda 2: si metteranno a confronto i risultati ottenuti dai tre algoritmi, accompagnati da un adeguato commento.\acapo

Mettendo a confronto i tre algoritmi è subito evidente come ci sia una evidente differenza di performance tra \texttt{Kruskal} e \texttt{Prim} con \texttt{NaiveKruskal}.

\image{0.9}{compare_n}{Performance dei tre algoritmi a confronto} 

Inizialmente \texttt{Prim} e \texttt{NaiveKruskal} delineano un andamento simile, tuttavia è possibile osservare come, già a partire dai grafi con 20K nodi, i tempi impiegati da \texttt{NaiveKruskal} risultino essere significativamente superiori a quelli di \texttt{Prim}. Questo perché, all'aumentare dei nodi, il tempo di esecuzione dell'algoritmo \texttt{NaiveKruskal}, con complessità \comp{mn}, cresce asintoticamente più rapidamente di quello impiegato dall'algoritmo \texttt{Prim}, avente complessità \comp{m\log n}.\eqcapo

D'altra parte, l'algoritmo \texttt{Kruskal} riesce sempre ad impiegare meno tempo degli altri algoritmi: è infatti possibile vedere come, anche nei grafi da 100K nodi, impieghi solamente $\sim$0.1-0.2 secondi per calcolare il costo del MST. Questo potrebbe essere considerato un comportamento sospetto, poiché i due algoritmi \texttt{Prim} e \texttt{Kruskal} impiegano tempi d'esecuzione molto distanti pur avendo la stessa complessità asintotica. \\
In realtà, il concetto di complessità asintotica è molto diverso dal \emph{computational time}: i due concetti non vanno confusi ma considerati come due entità distinte. Il primo concetto, infatti, assume:
\begin{itemize}
	\item che ogni tipologia di istruzione abbia il medesimo tempo;
	\item che questo tempo rimanga costante nel tempo.
\end{itemize} 
Infatti, per il calcolo della complessità, viene assunto che ogni istruzione elementare abbia complessità \comp{1}. Si tratta, però, di un calcolo puramente teorico che serve per dare l'idea del comportamento asintotico dell'algoritmo all'aumentare della dimensione dell'input.\eqcapo
Questo concetto si scontra con il \emph{computational time}: nella realtà, infatti, ogni istruzione atomica può impiegare tempi molto diversi. Inoltre, non si tiene conto di alcune operazioni necessarie che il processore deve compiere, come ad esempio il \emph{context switch} della cache. Anche la politica di scrittura utilizzata per mantenere la coerenza della memoria, può occupare il bus di sistema e, conseguentemente, causare ritardi nei tempi di esecuzione.\acapo

Un altro fatto che potrebbe destare qualche perplessità è la sostanziale differenza tra le tempistiche misurare dal calcolatore di Federico rispetto a quelle misurate dal calcolatore di Nicola. Anche qui si può vedere come, in realtà, il \emph{computational time} non sia un indicatore valido della complessità di un algoritmo: si può vedere come i tempi misurati utilizzando un determinato processore siano quasi la metà rispetto ai tempi misurati con un altro processore, a riprova del fatto che complessità asintotica e \emph{computational time} siano due cose completamente diverse.
